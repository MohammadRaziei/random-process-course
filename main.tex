\documentclass[12pt,onecolumn,a4paper]{article}
\usepackage{amsthm,amsmath,amssymb,bm}
\usepackage{epsfig,graphicx,subcaption}
\usepackage{float}
\usepackage{color,xcolor}
\usepackage{fmtcount}
\usepackage{placeins}
\usepackage{adjustbox}
\usepackage{tikz}
\usepackage{pgfplots}
\pgfplotsset{compat=1.18}


\usepackage{csvsimple}
\usepackage[top=1in, left=1in, right=1in, bottom=1in]{geometry}
\usepackage{nicefrac}
\usepackage{fancyhdr}
\usepackage{listings}
\usepackage{tabularx, booktabs, makecell}
\usepackage{hyperref,url}
\usepackage{listings}
\usepackage{natbib}
\usepackage{siunitx}

\usepackage{multirow}



\renewcommand{\baselinestretch}{1.5} 


\makeatletter
\let\old@lstKV@SwitchCases\lstKV@SwitchCases
\def\lstKV@SwitchCases#1#2#3{}
\makeatother
\usepackage{lstlinebgrd}
\makeatletter
\let\lstKV@SwitchCases\old@lstKV@SwitchCases

\lst@Key{numbers}{none}{%
	\def\lst@PlaceNumber{\lst@linebgrd}%
	\lstKV@SwitchCases{#1}%
	{none:\\%
		left:\def\lst@PlaceNumber{\llap{\normalfont
				\lst@numberstyle{\thelstnumber}\kern\lst@numbersep}\lst@linebgrd}\\%
		right:\def\lst@PlaceNumber{\rlap{\normalfont
				\kern\linewidth \kern\lst@numbersep
				\lst@numberstyle{\thelstnumber}}\lst@linebgrd}%
	}{\PackageError{Listings}{Numbers #1 unknown}\@ehc}}
\makeatother
\newcounter{subListing}[subfigure]


\definecolor{codegreen}{rgb}{0,0.6,0}
\definecolor{codegray}{rgb}{0.5,0.5,0.5}
\definecolor{codepurple}{rgb}{0.58,0,0.82}
\definecolor{mygreen}{RGB}{28,172,0} 
\definecolor{mylilas}{RGB}{170,55,241}
\definecolor{backcolour}{rgb}{1,1,0.98}

\lstset{language=MATLAB,%
	backgroundcolor=\color{backcolour},   
	commentstyle=\color{codegreen},
	keywordstyle=\color{blue},
	numberstyle=\tiny\color{codegray},
	stringstyle=\color{codepurple},
	basicstyle=\tt\scriptsize,
	frame = LBtr,
	%frameround=T,
	rulecolor=\color{gray},
	showstringspaces=false,
	numbers=left,%
	numberstyle={\tiny\color{gray}},
	numbersep=8pt,
	breaklines=true,
	%postbreak=\mbox{\textcolor{yellow}{$\hookrightarrow$}\space},
	tabsize=2,
	escapechar=`,
	xleftmargin=1.8 em, 
	framexleftmargin=2em,
}

\newcommand*{\transpose}{{\mkern-1.5mu\mathsf{T}}}


\usepackage{titlesec}
\titleformat{\section}[block]
{\titlerule\addvspace{4pt}\normalfont\fontsize{14}{16}\bfseries}
{\thesection\enspace}{0pt}{}[\vspace{2pt}\titlerule]


\usepackage{xepersian}
\settextfont{XB Niloofar}


\newcommand\question{
	\section{پاسخ سوال \tartibi{section}}
}


\author{
	محمد رضیئی فیجانی
}
\title{
	پاسخ تمرینات کامپیوتری سری اول فرایند تصادفی
}
\date{\today}


\newcounter{rownum} % Define a counter for the row number

\newcommand\makeresultcellNv[4]{
	\(
	\begin{matrix}
		a_{\text{est}} = 
		\begin{pmatrix}
			\csvreader[head=false, late after line=\\]{Q2/results/a-est-#2-N#3-v#4.csv}{}%
			{\num{\csvcoli}}
		\end{pmatrix}
		\\
		\text{#1} =
		\csvreader[head=false, 
		before reading=\setcounter{rownum}{0}, after line=\stepcounter{rownum} 
		]{Q2/results/a-est-#2-N#3-v#4-err.csv}{}%
		{\ifnum\therownum=0 \csvcoli \fi}
		\\
		\text{LSE} = 
		\csvreader[head=false, 
		before reading=\setcounter{rownum}{0}, after line=\stepcounter{rownum} 
		]{Q2/results/a-est-#2-N#3-v#4-err.csv}{}%
		{\ifnum\therownum=2 \num{\csvcoli} \fi}
		\\
		\nicefrac{\text{LSE}}{N} = 
		\csvreader[head=false, 
		before reading=\setcounter{rownum}{0}, after line=\stepcounter{rownum} 
		]{Q2/results/a-est-#2-N#3-v#4-err.csv}{}%
		{\ifnum\therownum=1 \num{\csvcoli} \fi}
	\end{matrix}
	\)
}




\begin{document}
	
	
	% Set the page style to "fancy"...
	\pagestyle{fancy}
	%... then configure it.
	%		\fancyhead{} % clear all header fields
	%		\fancyhead[RO,LE]{\textbf{The performance of new graduates}}
	%		\fancyfoot{} % clear all footer fields
	%		\fancyfoot[LE,RO]{\thepage}
	%		\fancyfoot[LO,CE]{From: K. Grant}
	%		\fancyfoot[CO,RE]{To: Dean A. Smith}
	\maketitle
	
	\section*{مقدمه}
	تمامی تمرینات در 
	\lr{Matlab R2024a}
	\cite{matlab2024a}
	توسعه پیدا کرده است.
	
	%%%%%%%%%%%%%%%%%%%%%%%%%%%%%%%%%%%%%%%%%%%%%%%%%%%%%%%%%%%%%%%%%%%%%
	\FloatBarrier
	\question%{خودهمبستگی و همبستگی متقابل}
	%%%%%%%%%%%%%%%%%%%%%%%%%%%%%%%%%%%%%%%%%%%%%%%%%%%%%%%%%%%%%%%%%%%%%
	
	%%%%%%%%%%%%%%%%%%%%%%%%%%%%%%%%%%%%%%%%%%%%%%%%%%%%%%%%%%%%%%%%%%%%%
	\FloatBarrier
	\subsection{قسمت الف:}
	
	فرض می‌کنیم \( x[n] \) یک نویز سفید گوسی با میانگین صفر و واریانس \( \sigma_x^2 = 1 \) است. خودهمبستگی این سیگنال به صورت زیر تعریف می‌شود:
	\begin{equation}
		R_{xx}[k] = \sigma_x^2 \delta[k] = \delta[k]
	\end{equation}
	
	
	از طرف دیگر، سیگنال \( y[n] \) با رابطه‌ی زیر به دست می‌آید:
	\begin{equation}
		y[n] = x[n] - 0.8 \, x[n-1] + 0.6 \, x[n-2]
	\end{equation}
	
	خودهمبستگی \( y[n] \) و همبستگی متقابل بین \( x[n] \) و \( y[n] \) به ترتیب با روابط زیر محاسبه می‌شود:
	
	\begin{equation}
		R_{yy}[k] = E[y[n] \, y[n+k]]
	\end{equation}
	
	\begin{equation}
		R_{xy}[k] = E[x[n] \, y[n+k]]
	\end{equation}
	
	با فرض اینکه \( x[n] \) نویز سفید گوسی است، این روابط به صورت تئوری و با استفاده از کانولوشن قابل محاسبه هستند. نمودارهای زیر نتایج تئوری محاسبات \( R_{xy} \) و \( R_{yy} \) را نشان می‌دهند.
	
	\begin{figure}[H]
		\centering
		\begin{subfigure}{.45\linewidth}
			\centering
			\includegraphics[width=\linewidth]{Q1/results/Rxy}
			\caption{$R_{xy}$}
			\label{fig:rxy}
		\end{subfigure}
		\hfill
		\begin{subfigure}{.45\linewidth}
			\centering
			\includegraphics[width=\linewidth]{Q1/results/Ryy}
			\caption{$R_{yy}$}
			\label{fig:ryy}
		\end{subfigure}
		\caption{خودهمبستگی و همبستگی متقابل به صورت تئوری}
	\end{figure}
	
	%%%%%%%%%%%%%%%%%%%%%%%%%%%%%%%%%%%%%%%%%%%%%%%%%%%%%%%%%%%%%%%%%%%%%
	\FloatBarrier
	\subsection{قسمت ب:}
	
	
	در این بخش، \( N \) نمونه از سیگنال \( x[n] \) را تولید کرده و به ازای \( N = 100 \) و \( N = 10000 \)، توابع خودهمبستگی و همبستگی متقابل را محاسبه و رسم می‌کنیم. برای تخمین خودهمبستگی از تابع `xcorr` استفاده می‌شود.
	
	\begin{latin}
		\begin{lstlisting}[language=Matlab]
			% Autocorrelation of x[n]
			R_x = xcorr(x, 'biased');
			
			% Autocorrelation of y[n]
			R_y = xcorr(y, 'biased');
			
			% Cross-correlation between x[n] and y[n]
			R_xy = xcorr(x, y, 'biased');
		\end{lstlisting}
	\end{latin}
	
	در نمودارهای زیر، نتایج تخمینی توابع خودهمبستگی و همبستگی متقابل نمایش داده شده است.
	
	\begin{figure}
		\centering
		\includegraphics[width=\linewidth]{Q1/results/xcorr-N100}
		\caption{خودهمبستگی و همبستگی متقابل برای \( N = 100 \)}
		\label{fig:xcorr-n100}
	\end{figure}
	
	\begin{figure}
		\centering
		\includegraphics[width=\linewidth]{Q1/results/xcorr-N10000}
		\caption{خودهمبستگی و همبستگی متقابل برای \( N = 10000 \)}
		\label{fig:xcorr-n10000}
	\end{figure}
	
	%%%%%%%%%%%%%%%%%%%%%%%%%%%%%%%%%%%%%%%%%%%%%%%%%%%%%%%%%%%%%%%%%%%%%
	\FloatBarrier
	\subsection{قسمت پ:}
	
	
	در این بخش، ابتدا تابع چگالی طیف توان \( x[n] \) و \( y[n] \) را به صورت تئوری محاسبه می‌کنیم.
	
	\subsubsection{تابع چگالی طیف توان برای \( x[n] \)}
	چون \( x[n] \) یک نویز سفید گوسی است:
	
	\begin{equation}
		S_{xx}(f) = \sigma_x^2 = 1
	\end{equation}
	
	\subsubsection{تابع چگالی طیف توان برای \( y[n] \)}
	تابع چگالی طیف توان برای \( y[n] \) با استفاده از بزرگی پاسخ فرکانسی سیستم محاسبه می‌شود:
	
	\begin{equation}
		S_{yy}(f) = |H(f)|^2 \cdot S_{xx}(f)
	\end{equation}
	
	که در آن:
	
	\begin{equation}
		H(f) = 1 - 0.8 \, e^{-j 2\pi f} + 0.6 \, e^{-j 4\pi f}
	\end{equation}
	
	و:
	
	\begin{equation}
		|H(f)| = \left| 1 - 0.8 \, e^{-j 2\pi f} + 0.6 \, e^{-j 4\pi f} \right|
	\end{equation}
	
	تابع چگالی طیف توان تخمینی نیز با استفاده از روش ولچ (Welch) محاسبه می‌شود:
	
	\begin{latin}
		\begin{lstlisting}
			% Calculate the Power Spectral Density using Welch's method
			[PSD_y, f] = pwelch(y, window_length, overlap, N, 1, 'centered');
		\end{lstlisting}
	\end{latin}
	
	\begin{figure}[H]
		\centering
		\includegraphics[width=\linewidth]{Q1/results/psd-N100}
		\caption{تابع چگالی طیف توان برای \( N = 100 \)}
		\label{fig:psd-n100}
	\end{figure}
	
	\begin{figure}[H]
		\centering
		\includegraphics[width=\linewidth]{Q1/results/psd-N10000}
		\caption{تابع چگالی طیف توان برای \( N = 10000 \)}
		\label{fig:psd-n10000}
	\end{figure}
	
	
	
	
	
	
	%%%%%%%%%%%%%%%%%%%%%%%%%%%%%%%%%%%%%%%%%%%%%%%%%%%%%%%%%%%%%%%%%%%%%
	\FloatBarrier
	\question%{خودهمبستگی و همبستگی متقابل}
	%%%%%%%%%%%%%%%%%%%%%%%%%%%%%%%%%%%%%%%%%%%%%%%%%%%%%%%%%%%%%%%%%%%%%
	
	%%%%%%%%%%%%%%%%%%%%%%%%%%%%%%%%%%%%%%%%%%%%%%%%%%%%%%%%%%%%%%%%%%%%%
	\FloatBarrier
	\subsection{قسمت الف:}
	
	
	
	\begin{equation}
		y[n] = 0.5 \, y[n-1] + 0.8 \, x[n] - 0.6 \, x[n-1] + v[n]
	\end{equation}
	
	
	\begin{equation}
		\bm{u}_n = \begin{bmatrix}
			y[n-1] \\
			x[n] \\
			x[n-1]
		\end{bmatrix}
	\end{equation}
	
	
	رابطه به‌صورت زیر است:
	
	\begin{equation}
		(\bm{u}_n)^\transpose \bm{a}
		= y[n]
	\end{equation}
	
	که در آن $\bm{a}$ در این مسیله به شکل زیر تعریف شده است:
	\begin{equation}
		\bm{a} = 
		\begin{bmatrix}
			a_{-1} \\
			a_{0} \\
			a_{1}
		\end{bmatrix} =
		\begin{bmatrix}
			0.5 \\
			0.8 \\
			-0.6
		\end{bmatrix}
	\end{equation}
	
	
	\begin{equation}
		\bm{y} = \bm{U} \bm{a}
	\end{equation}
	
	
	\begin{equation}
		\bm{U} = 
		\begin{bmatrix}
			y[0] & x[1] & x[0] \\
			y[1] & x[2] & x[1] \\
			y[2] & x[3] & x[2] \\
			\vdots & \vdots & \vdots \\
			y[N-2] & x[N-1] & x[N-2]
		\end{bmatrix}
		; \quad
		\bm{y} = 
		\begin{bmatrix}
			y[1] \\
			y[2] \\
			\vdots \\
			y[N-1]
		\end{bmatrix}
	\end{equation}
	
	
	
	
	
	
	
	
	
	
	
	
	
	
	
	
	
	
	
	
	
	
	
	
	
	
	
	
	هدف ما کمینه‌سازی عبارت زیر است:
	\begin{equation}
		\min_{\bm{a}} \big\| \bm{y} - \bm{U} \bm{a} \big\|_2^2
	\end{equation}
	
	
	\begin{equation}
		\bm{a}_\text{est} =
		 (\bm{U}^\transpose \bm{U})^{-1} \bm{U}^\transpose \bm{y} = 
		 \bm{U}^\dagger \bm{y}
	\end{equation}
	
	
	\begin{equation}
		\bm{a}_\text{est} =
		\begin{bmatrix}
			% Read the CSV file and insert the values into the matrix
			\csvreader[head=false, late after line=\\]{Q2/results/a-est-3.csv}{}%
			{\csvcoli}
		\end{bmatrix}
	\end{equation}
	
	
	
	
	
	
	
	
	
	
	
	
	\subsubsection{الف۱}
	
	\subsubsection{الف۲}
	
	در صورتی که تصور کنیم،  بردار مجهولات جدید مطابق زیر باشد:
	\begin{equation}
		\bm{a} = \begin{bmatrix} a_{-2} \\ a_{-1} \\ a_0 \\ a_1 \\ a_2 \end{bmatrix}
	\end{equation}
	
	رابطه‌ی ورودی-خروجی به صورت زیر است:
	\begin{equation}
		y[n] = a_{-2} \, y[n-2] + a_{-1} \, y[n-1] + a_0 \, x[n] + a_1 \, x[n-1] + a_2 \, x[n-2]
	\end{equation}
	
	
	
	\begin{equation}
		\bm{U} = 
		\begin{bmatrix}
			y[0] & y[1] & x[2] & x[1] & x[0] \\
			y[1] & y[2] & x[3] & x[2] & x[1] \\
			y[2] & y[3] & x[4] & x[3] & x[2] \\
			\vdots & \vdots & \vdots & \vdots & \vdots \\
			y[N-3] & y[N-2] & x[N-1] & x[N-2] & x[N-3]
		\end{bmatrix}
		; \quad
		\bm{y} = 
		\begin{bmatrix}
			y[2] \\
			y[3] \\
			\vdots \\
			y[N-2]
		\end{bmatrix}
	\end{equation}
	
	
	
	\begin{equation}
		\bm{a}_\text{est} =
		\begin{bmatrix}
			% Read the CSV file and insert the values into the matrix
			\csvreader[head=false, late after line=\\]{Q2/results/a-est-5.csv}{}%
			{\csvcoli}
		\end{bmatrix}
	\end{equation}
	
	
	\subsubsection{الف۳}
	
	\begin{equation}
		y[n] = \sum_{i=0}^M a_i x[n-i]
	\end{equation}
	
	
	\begin{equation}
		\bm{a} = \begin{bmatrix}  a_0 \\ a_1 \\ \vdots \\ a_M \end{bmatrix}
	\end{equation}
	
	
	\begin{equation}
		\bm{U} = 
		\begin{bmatrix}
			x[M] & x[M-1] & \cdots & x[1] & x[0] \\
			x[M+1] & x[M] & \cdots & x[2] & x[1] \\
			x[M+2] & x[M+1] & \cdots & x[3] & x[2] \\
			\vdots & \vdots &  & \vdots & \vdots \\
			x[N-1] & x[N-2] & \cdots & x[N-M] & x[N-M-1]
		\end{bmatrix}_{(N-M)\times(M+1)}
	\end{equation}
	
	\begin{equation}
		\bm{y} = 
		\begin{bmatrix}
			y[M] \\
			y[M+1] \\
			\vdots \\
			y[N-1]
		\end{bmatrix}_{(N-M)\times1}
	\end{equation}
	
	
	
	\begin{table}
		\centering
		\caption{}
		\begin{LTR}
			\begin{tabular}{|c||c|c|c|}
				\hline
				\(M\)& 5 & 10 & 15 \\\hline
				\(	\bm{a}_\text{est} \) & 
				\(
				\begin{matrix}
					\csvreader[head=false, late after line=\\]{Q2/results/a-est-ma-5.csv}{}%
					{\num{\csvcoli}}
				\end{matrix}
				\) &
				\(
				\begin{matrix}
					\csvreader[head=false, late after line=\\]{Q2/results/a-est-ma-10.csv}{}%
					{\num{\csvcoli}}
				\end{matrix}
				\) & 
				\(
				\begin{matrix}
					\csvreader[head=false, late after line=\\]{Q2/results/a-est-ma-15.csv}{}%
					{\num{\csvcoli}}
				\end{matrix}
				\)
				\\\hline
			\end{tabular}
		\end{LTR}
	\end{table}
	
	
	
	
	
	
	
	
	\begin{figure}[H]
		\centering
		
		
		\begin{tikzpicture}
			\pgfmathsetmacro{\barwidth}{.2}  % Set to a high initial value
			\pgfmathsetmacro{\minMetric}{-.4}  % Set to a high initial value
			\pgfmathsetmacro{\maxMetric}{1}     % Set to a low initial value
			
			\pgfmathsetmacro{\ymax}{\maxMetric}
			\pgfmathsetmacro{\ymin}{\minMetric}
			\begin{axis}[
				xlabel={$\bm{a}$},
				ylabel={value},
				title={$\bm{a}$ for different value of $M$},
				grid=both,
				width=\linewidth, height=.5\linewidth,
				ymin=\ymin, ymax=\ymax,
				xmin=-1, xmax=16,
				xtick={0,1,...,15},
				xticklabels={% تنظیم برچسب‌های محور x به‌صورت a_i
					$a_0$, $a_1$, $a_2$, $a_3$, $a_4$, $a_5$, $a_6$, $a_7$, $a_8$, $a_9$, $a_{10}$, $a_{11}$, $a_{12}$, $a_{13}$, $a_{14}$, $a_{15}$
				},
				]
				\csvreader[head=false, 
				before reading=\setcounter{rownum}{0}, % تنظیم شمارنده به صفر
				after line=\stepcounter{rownum} % افزایش شمارنده در هر خط
				]{Q2/results/a-est-ma-5.csv}{}%
				{ \addplot+[
					style=solid,
					color=blue,
					fill=blue,
					ybar, bar width=0.2cm,
					mark=none 
					] coordinates {(\therownum-\barwidth,\csvcoli)};}
				
				\csvreader[head=false, 
				before reading=\setcounter{rownum}{0}, 
				after line=\stepcounter{rownum} 
				]{Q2/results/a-est-ma-10.csv}{}%
				{ \addplot+[
					style=solid,
					color=red,
					fill=red,
					ybar, bar width=\barwidth cm,
					mark=none 
					] coordinates {(\therownum,\csvcoli)};}
				
				\csvreader[head=false, 
				before reading=\setcounter{rownum}{0}, 
				after line=\stepcounter{rownum} 
				]{Q2/results/a-est-ma-15.csv}{}%
				{ \addplot+[
					style=solid,
					color=green,
					fill=green,
					ybar, bar width=\barwidth cm,
					mark=none 
					] coordinates {(\therownum+\barwidth,\csvcoli)};}
			\end{axis}
			
			% ایجاد باکس حامل راهنما
			\node[draw, color=gray, anchor=north west, fill=white] (legendbox) at (0.73\linewidth, 0.39\linewidth) {%
				\begin{tikzpicture}
					% راهنمای M=5
					\node[fill=blue, draw=black, minimum width=0.4cm, minimum height=0.25cm] 
					at (0,0+0.13) {};
					\node[anchor=west, color=black] at (0.5,0) {$M=5$};
					
					% راهنمای M=10
					\node[fill=red, draw=black, minimum width=0.4cm, minimum height=0.25cm] 
					at (0,-0.5+.13) {};
					\node[anchor=west, color=black] at (0.5,-0.5) {$M=10$};
					
					% راهنمای M=15
					\node[fill=green, draw=black, minimum width=0.4cm, minimum height=0.25cm] 
					at (0,-1+.13) {};
					\node[anchor=west, color=black] at (0.5,-1) {$M=15$};
				\end{tikzpicture}
			};
			
			
		\end{tikzpicture}
		
		
		
	\end{figure}
	
	
	
	
	
	
	
	\subsection{قسمت ب}
	
	
	\subsubsection{ب۱}
	
	
	\begin{table}[H]
	\centering
	\caption{}
	\begin{LTR}
		\begin{tabular}{l||c|c|c|}
			\cline{2-4}
			& $N = 100$ & $N = 1000$  & $N = 10000$ \\\hline\hline 
			\!\!\!\vrule\ \	$\sigma^2_v = 0.1$ &
			\makeresultcellNv{$\big\|{\bm{a}} - \bm{a}_{\text{est}}\big\|_2$}{3}{100}{0.10}
			&
			\makeresultcellNv{$\big\|{\bm{a}} - \bm{a}_{\text{est}}\big\|_2$}{3}{1000}{0.10}
			&
			\makeresultcellNv{$\big\|{\bm{a}} - \bm{a}_{\text{est}}\big\|_2$}{3}{10000}{0.10}
			\\\hline
			\!\!\!\vrule\ \	$\sigma^2_v = 0.01$ &
			\makeresultcellNv{$\big\|{\bm{a}} - \bm{a}_{\text{est}}\big\|_2$}{3}{100}{0.01}
			&
			\makeresultcellNv{$\big\|{\bm{a}} - \bm{a}_{\text{est}}\big\|_2$}{3}{1000}{0.01}
			&
			\makeresultcellNv{$\big\|{\bm{a}} - \bm{a}_{\text{est}}\big\|_2$}{3}{10000}{0.01}
			\\\hline
		\end{tabular}
	\end{LTR}
\end{table}

	\begin{equation}
	\nicefrac{\text{LSE}}{N} \approx \sigma^2_v
\end{equation}





	
	\subsubsection{ب2}
	
	
	
	\begin{table}[H]
		\centering
		\caption{}
		\begin{LTR}
			\begin{tabular}{l||c|c|c|}
				\cline{2-4}
				& $N = 100$ & $N = 1000$  & $N = 10000$ \\\hline\hline 
				\!\!\!\vrule\ \	$\sigma^2_v = 0.1$ &
				\makeresultcellNv{$\big\|\tilde{\bm{a}} - \bm{a}_{\text{est}}\big\|_2$}{5}{100}{0.10}
				&
				\makeresultcellNv{$\big\|\tilde{\bm{a}} - \bm{a}_{\text{est}}\big\|_2$}{5}{1000}{0.10}
				&
				\makeresultcellNv{$\big\|\tilde{\bm{a}} - \bm{a}_{\text{est}}\big\|_2$}{5}{10000}{0.10}
				\\\hline
				\!\!\!\vrule\ \	$\sigma^2_v = 0.01$ &
				\makeresultcellNv{$\big\|\tilde{\bm{a}} - \bm{a}_{\text{est}}\big\|_2$}{5}{100}{0.01}
				&
				\makeresultcellNv{$\big\|\tilde{\bm{a}} - \bm{a}_{\text{est}}\big\|_2$}{5}{1000}{0.01}
				&
				\makeresultcellNv{$\big\|\tilde{\bm{a}} - \bm{a}_{\text{est}}\big\|_2$}{5}{10000}{0.01}
				\\\hline
			\end{tabular}
		\end{LTR}
	\end{table}
	
		\begin{equation}
		\nicefrac{\text{LSE}}{N} \approx \sigma^2_v
	\end{equation}
	
	
	
	
	
	
	
	\subsubsection{ب۳}
	
	

	
	
	
	
	\newpage
	\section{پیوست‌ها}
	
	
	\subsection{محاسبه کمترین مربعات خطا (LSE)}
	
	
	
	هدف این بخش، پیدا کردن بردار پارامترهای تخمین‌گر خطی \( \bm{x} \) به‌گونه‌ای است که فاصله اقلیدسی بین بردار مشاهدات \( \bm{y} \) و مدل خطی \( \bm{A} \bm{x} \) کمینه شود.
	\begin{equation}
		\min_{\bm{x}} \| \bm{y} - \bm{A} \bm{x} \|_2^2
	\end{equation}
	
	که در آن \( \bm{y} \) بردار مشاهدات، \( \bm{A} \) ماتریس طراحی و \( \bm{x} \) بردار پارامترهای مجهول است. تابع هدف به صورت زیر تعریف می‌شود:
	\begin{equation}
		f(\bm{x}) = \| \bm{y} - \bm{A} \bm{x} \|_2^2 = (\bm{y} - \bm{A} \bm{x})^\transpose (\bm{y} - \bm{A} \bm{x})
	\end{equation}
	
	با باز کردن این تابع:
	\begin{equation}
		f(\bm{x}) = \bm{y}^\transpose \bm{y} - 2 \bm{x}^\transpose \bm{A}^\transpose \bm{y} + \bm{x}^\transpose \bm{A}^\transpose \bm{A} \bm{x}
	\end{equation}
	
	برای پیدا کردن نقطه‌ی بهینه، مشتق تابع \( f(\bm{x}) \) را نسبت به \( \bm{x} \) محاسبه کرده و آن را برابر صفر قرار می‌دهیم:
	\begin{equation}
		{\nabla} f(\bm{x}) = -2 \bm{A}^\transpose \bm{y} + 2 \bm{A}^\transpose \bm{A} \bm{x} = 0
	\end{equation}
	
	با ساده‌سازی، معادله‌ی نرمال به دست می‌آید:
	\begin{equation}
		\bm{A}^\transpose \bm{A} \bm{x} = \bm{A}^\transpose \bm{y}
	\end{equation}
	
	که با حل این معادله، بردار \( \bm{x} \) بهینه به صورت زیر خواهد بود:
	\begin{equation}
		\bm{x} = (\bm{A}^\transpose \bm{A})^{-1} \bm{A}^\transpose \bm{y}
	\end{equation}
	
	حال، بردار خطا را به دست می‌آوریم:
	\begin{equation}
		\bm{e} = \bm{y} - \bm{A} \bm{x} = \bm{y} - \bm{A} (\bm{A}^\transpose \bm{A})^{-1} \bm{A}^\transpose \bm{y}
	\end{equation}
	
	بنابراین، خطای کمترین مربعات برابر است با:
	\begin{equation}
		\| \bm{e} \|_2^2 = \| \bm{y} - \bm{A} \bm{x} \|_2^2
	\end{equation}
	
	با معرفی \( \bm{P} = \bm{A} (\bm{A}^\transpose \bm{A})^{-1} \bm{A}^\transpose \) به عنوان ماتریس پروجکشن، می‌توان خطای کمترین مربعات را به صورت زیر نوشت:
	\begin{equation}
		\| \bm{e} \|_2^2 = \| (\bm{I} - \bm{P}) \bm{y} \|_2^2
	\end{equation}
	
	که در آن، \( \bm{I} \) ماتریس واحد است و \( \bm{P} \) بردار \( \bm{y} \) را بر روی زیرفضای ستون‌های \( \bm{A} \) تصویر می‌کند. در نتیجه، کمترین مربعات خطا به صورت زیر خواهد بود:
	\begin{equation}
		\text{\lr{Least Square Error}} = \| (\bm{I} - \bm{P}) \bm{y} \|_2^2
	\end{equation}
	
	
	
	
	\subsection{تعریف سودو اینورس (Pseudo-inverse)}
	
	
	
	\noindent
	سودو اینورس یا \textit{معکوس مور-پنروز}
	\LTRfootnote{Moore-Penrose}
	، یک تعمیم از معکوس ماتریس است که می‌تواند برای هر ماتریسی، چه مربعی و چه غیرمربعی، چه منفرد و چه غیرمنفرد، تعریف شود. برای یک ماتریس \( \bm{A} \) با ابعاد \( m \times n \)، سودو اینورس آن که با \( \bm{A}^\dagger \) نشان داده می‌شود، یکتاست و شرایط زیر را ارضا می‌کند:
	
	\begin{align}
		\bm{A} \, \bm{A}^\dagger \, \bm{A} &= \bm{A}
		\\
		\bm{A}^\dagger \, \bm{A} \, \bm{A}^\dagger &= \bm{A}^\dagger
		\\
		(\bm{A} \, \bm{A}^\dagger)^\transpose &= \bm{A} \, \bm{A}^\dagger
		\\
		(\bm{A}^\dagger \, \bm{A})^\transpose &= \bm{A}^\dagger \, \bm{A}
	\end{align}
	
	
	
	\noindent
	اگر ماتریس \( \bm{A} \) به صورت کامل رتبه باشد، سودو اینورس آن به شکل زیر محاسبه می‌شود:
	
	\noindent
	- اگر \( m \geq n \) (تعداد سطرها بیشتر از تعداد ستون‌ها باشد):
	\begin{equation}
		\bm{A}^\dagger = (\bm{A}^\transpose \, \bm{A})^{-1} \, \bm{A}^\transpose
	\end{equation}
	
	\noindent
	- اگر \( m < n \) (تعداد ستون‌ها بیشتر از تعداد سطرها باشد):
	\begin{equation}
		\bm{A}^\dagger = \bm{A}^\transpose \, (\bm{A} \, \bm{A}^\transpose)^{-1}
	\end{equation}
	
	\noindent
	در حالت کلی و برای ماتریس‌هایی که رتبه کامل ندارند، سودو اینورس با استفاده از تجزیه مقدار منفرد (SVD) به صورت زیر محاسبه می‌شود:
	\begin{equation}
		\bm{A} = \bm{U} \, \bm{\Sigma} \, \bm{V}^\transpose \implies \bm{A}^\dagger = \bm{V} \, \bm{\Sigma}^\dagger \, \bm{U}^\transpose
	\end{equation}
	
	که در آن:
	\( \bm{U} \) و \( \bm{V} \) ماتریس‌های متعامد هستند.
	\( \bm{\Sigma} \) ماتریس قطری شامل مقادیر منفرد است.
	\( \bm{\Sigma}^\dagger \) با معکوس کردن مقادیر غیرصفر در \( \bm{\Sigma} \) به دست می‌آید.
	
	
	
	%%%%%%%%%%%%%%%%%%%%%%%%%%%%%%%%%%%%%%%%%%%%%%%%%%%%%%%%%%%%%%%%%%%%
	\newpage
	\bibliographystyle{plainnat-fa}
	%\nocite{*}
	\bibliography{references}
	
	
\end{document}